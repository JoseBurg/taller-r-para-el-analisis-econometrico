% Options for packages loaded elsewhere
\PassOptionsToPackage{unicode}{hyperref}
\PassOptionsToPackage{hyphens}{url}
%
\documentclass[
  ignorenonframetext,
]{beamer}
\usepackage{pgfpages}
\setbeamertemplate{caption}[numbered]
\setbeamertemplate{caption label separator}{: }
\setbeamercolor{caption name}{fg=normal text.fg}
\beamertemplatenavigationsymbolsempty
% Prevent slide breaks in the middle of a paragraph
\widowpenalties 1 10000
\raggedbottom
\setbeamertemplate{part page}{
  \centering
  \begin{beamercolorbox}[sep=16pt,center]{part title}
    \usebeamerfont{part title}\insertpart\par
  \end{beamercolorbox}
}
\setbeamertemplate{section page}{
  \centering
  \begin{beamercolorbox}[sep=12pt,center]{part title}
    \usebeamerfont{section title}\insertsection\par
  \end{beamercolorbox}
}
\setbeamertemplate{subsection page}{
  \centering
  \begin{beamercolorbox}[sep=8pt,center]{part title}
    \usebeamerfont{subsection title}\insertsubsection\par
  \end{beamercolorbox}
}
\AtBeginPart{
  \frame{\partpage}
}
\AtBeginSection{
  \ifbibliography
  \else
    \frame{\sectionpage}
  \fi
}
\AtBeginSubsection{
  \frame{\subsectionpage}
}
\usepackage{amsmath,amssymb}
\usepackage{iftex}
\ifPDFTeX
  \usepackage[T1]{fontenc}
  \usepackage[utf8]{inputenc}
  \usepackage{textcomp} % provide euro and other symbols
\else % if luatex or xetex
  \usepackage{unicode-math} % this also loads fontspec
  \defaultfontfeatures{Scale=MatchLowercase}
  \defaultfontfeatures[\rmfamily]{Ligatures=TeX,Scale=1}
\fi
\usepackage{lmodern}
\usetheme[]{Madrid}
\ifPDFTeX\else
  % xetex/luatex font selection
\fi
% Use upquote if available, for straight quotes in verbatim environments
\IfFileExists{upquote.sty}{\usepackage{upquote}}{}
\IfFileExists{microtype.sty}{% use microtype if available
  \usepackage[]{microtype}
  \UseMicrotypeSet[protrusion]{basicmath} % disable protrusion for tt fonts
}{}
\makeatletter
\@ifundefined{KOMAClassName}{% if non-KOMA class
  \IfFileExists{parskip.sty}{%
    \usepackage{parskip}
  }{% else
    \setlength{\parindent}{0pt}
    \setlength{\parskip}{6pt plus 2pt minus 1pt}}
}{% if KOMA class
  \KOMAoptions{parskip=half}}
\makeatother
\usepackage{xcolor}
\newif\ifbibliography
\usepackage{color}
\usepackage{fancyvrb}
\newcommand{\VerbBar}{|}
\newcommand{\VERB}{\Verb[commandchars=\\\{\}]}
\DefineVerbatimEnvironment{Highlighting}{Verbatim}{commandchars=\\\{\}}
% Add ',fontsize=\small' for more characters per line
\usepackage{framed}
\definecolor{shadecolor}{RGB}{248,248,248}
\newenvironment{Shaded}{\begin{snugshade}}{\end{snugshade}}
\newcommand{\AlertTok}[1]{\textcolor[rgb]{0.94,0.16,0.16}{#1}}
\newcommand{\AnnotationTok}[1]{\textcolor[rgb]{0.56,0.35,0.01}{\textbf{\textit{#1}}}}
\newcommand{\AttributeTok}[1]{\textcolor[rgb]{0.13,0.29,0.53}{#1}}
\newcommand{\BaseNTok}[1]{\textcolor[rgb]{0.00,0.00,0.81}{#1}}
\newcommand{\BuiltInTok}[1]{#1}
\newcommand{\CharTok}[1]{\textcolor[rgb]{0.31,0.60,0.02}{#1}}
\newcommand{\CommentTok}[1]{\textcolor[rgb]{0.56,0.35,0.01}{\textit{#1}}}
\newcommand{\CommentVarTok}[1]{\textcolor[rgb]{0.56,0.35,0.01}{\textbf{\textit{#1}}}}
\newcommand{\ConstantTok}[1]{\textcolor[rgb]{0.56,0.35,0.01}{#1}}
\newcommand{\ControlFlowTok}[1]{\textcolor[rgb]{0.13,0.29,0.53}{\textbf{#1}}}
\newcommand{\DataTypeTok}[1]{\textcolor[rgb]{0.13,0.29,0.53}{#1}}
\newcommand{\DecValTok}[1]{\textcolor[rgb]{0.00,0.00,0.81}{#1}}
\newcommand{\DocumentationTok}[1]{\textcolor[rgb]{0.56,0.35,0.01}{\textbf{\textit{#1}}}}
\newcommand{\ErrorTok}[1]{\textcolor[rgb]{0.64,0.00,0.00}{\textbf{#1}}}
\newcommand{\ExtensionTok}[1]{#1}
\newcommand{\FloatTok}[1]{\textcolor[rgb]{0.00,0.00,0.81}{#1}}
\newcommand{\FunctionTok}[1]{\textcolor[rgb]{0.13,0.29,0.53}{\textbf{#1}}}
\newcommand{\ImportTok}[1]{#1}
\newcommand{\InformationTok}[1]{\textcolor[rgb]{0.56,0.35,0.01}{\textbf{\textit{#1}}}}
\newcommand{\KeywordTok}[1]{\textcolor[rgb]{0.13,0.29,0.53}{\textbf{#1}}}
\newcommand{\NormalTok}[1]{#1}
\newcommand{\OperatorTok}[1]{\textcolor[rgb]{0.81,0.36,0.00}{\textbf{#1}}}
\newcommand{\OtherTok}[1]{\textcolor[rgb]{0.56,0.35,0.01}{#1}}
\newcommand{\PreprocessorTok}[1]{\textcolor[rgb]{0.56,0.35,0.01}{\textit{#1}}}
\newcommand{\RegionMarkerTok}[1]{#1}
\newcommand{\SpecialCharTok}[1]{\textcolor[rgb]{0.81,0.36,0.00}{\textbf{#1}}}
\newcommand{\SpecialStringTok}[1]{\textcolor[rgb]{0.31,0.60,0.02}{#1}}
\newcommand{\StringTok}[1]{\textcolor[rgb]{0.31,0.60,0.02}{#1}}
\newcommand{\VariableTok}[1]{\textcolor[rgb]{0.00,0.00,0.00}{#1}}
\newcommand{\VerbatimStringTok}[1]{\textcolor[rgb]{0.31,0.60,0.02}{#1}}
\newcommand{\WarningTok}[1]{\textcolor[rgb]{0.56,0.35,0.01}{\textbf{\textit{#1}}}}
\usepackage{longtable,booktabs,array}
\usepackage{calc} % for calculating minipage widths
\usepackage{caption}
% Make caption package work with longtable
\makeatletter
\def\fnum@table{\tablename~\thetable}
\makeatother
\usepackage{graphicx}
\makeatletter
\def\maxwidth{\ifdim\Gin@nat@width>\linewidth\linewidth\else\Gin@nat@width\fi}
\def\maxheight{\ifdim\Gin@nat@height>\textheight\textheight\else\Gin@nat@height\fi}
\makeatother
% Scale images if necessary, so that they will not overflow the page
% margins by default, and it is still possible to overwrite the defaults
% using explicit options in \includegraphics[width, height, ...]{}
\setkeys{Gin}{width=\maxwidth,height=\maxheight,keepaspectratio}
% Set default figure placement to htbp
\makeatletter
\def\fps@figure{htbp}
\makeatother
\setlength{\emergencystretch}{3em} % prevent overfull lines
\providecommand{\tightlist}{%
  \setlength{\itemsep}{0pt}\setlength{\parskip}{0pt}}
\setcounter{secnumdepth}{-\maxdimen} % remove section numbering
\ifLuaTeX
  \usepackage{selnolig}  % disable illegal ligatures
\fi
\usepackage{bookmark}
\IfFileExists{xurl.sty}{\usepackage{xurl}}{} % add URL line breaks if available
\urlstyle{same}
\hypersetup{
  pdftitle={R para el Análisis Econométrico},
  pdfauthor={José Burgos \textbar{} ASOECO},
  hidelinks,
  pdfcreator={LaTeX via pandoc}}

\title{R para el Análisis Econométrico}
\subtitle{Módulo 1: Fundamentos Estadísticos en R}
\author{José Burgos \textbar{} ASOECO}
\date{07-09-2025}

\begin{document}
\frame{\titlepage}

\section{Parte I: Fundamentos de R y
RStudio}\label{parte-i-fundamentos-de-r-y-rstudio}

\begin{frame}{R y RStudio}
\phantomsection\label{r-y-rstudio}
\begin{block}{Bienvenida a R y RStudio}
\phantomsection\label{bienvenida-a-r-y-rstudio}
\begin{itemize}
\tightlist
\item
  ¿Qué es R? Un lenguaje de programación para computación estadística y
  gráficos.
\end{itemize}

\begin{itemize}
\item
  ¿Qué es RStudio? El Entorno de Desarrollo Integrado (IDE).
\item
  Filosofía de R: código abierto, extensible a través de paquetes y con
  una gran comunidad.
\end{itemize}
\end{block}
\end{frame}

\begin{frame}{R y RStudio}
\phantomsection\label{r-y-rstudio-1}
\begin{block}{Hacer las cosas bien desde el inicio}
\phantomsection\label{hacer-las-cosas-bien-desde-el-inicio}
\begin{itemize}
\item
  Buenas prácticas desde el inicio: Investigación Reproducible.
\item
  Introducción al concepto de proyectos en RStudio para mantener el
  trabajo organizado.
\item
  Entorno de trabajo y creación de un primer proyecto.
\end{itemize}
\end{block}

\begin{block}{Manejo del entorno y flujo de trabajo}
\phantomsection\label{manejo-del-entorno-y-flujo-de-trabajo}
\begin{itemize}
\tightlist
\item
  Scripts de R: Cómo escribir y ejecutar código desde un archivo .R.
\item
  Atajos de teclado clave en RStudio para agilizar el trabajo (ejecutar
  código, asignación, pipe).
\end{itemize}
\end{block}
\end{frame}

\section{Primeros pasos en la programación con
R}\label{primeros-pasos-en-la-programaciuxf3n-con-r}

\begin{frame}[fragile]{Operadores aritméticos}
\phantomsection\label{operadores-aritmuxe9ticos}
Tal como sugiere su nombre, estos operadores se emplea específicamente
para realizar cálculos aritméticos. Es factible efectuar estas
operaciones utilizando datos tanto enteros como numéricos.

\begin{longtable}[]{@{}llll@{}}
\toprule\noalign{}
Operador & Operación & Ejemplo & Resultado \\
\midrule\noalign{}
\endhead
\texttt{+} & Suma & \texttt{5+3} & \texttt{8} \\
\texttt{-} & Resta & \texttt{5-3} & \texttt{2} \\
\texttt{*} & Multiplicación & \texttt{5*3} & \texttt{15} \\
\texttt{/} & División & \texttt{5/3} & \texttt{1.666667} \\
\texttt{\^{}} & Potencia & \texttt{5\^{}3} & \texttt{125} \\
\texttt{\%\%} & División entera & \texttt{5\%\%3} & \texttt{2} \\
\bottomrule\noalign{}
\end{longtable}

El orden de las operaciones es similar al que se usa en matemáticas:
primero se resuelven las potencias, luego las multiplicaciones y
divisiones, y finalmente las sumas y restas.
\end{frame}

\begin{frame}[fragile]{Operadores relacionados}
\phantomsection\label{operadores-relacionados}
Se usan los operadores lógicos para hacer comparaciones y su resultado
siempre será \texttt{TRUE} o \texttt{FALSE} (verdadero o flaso,
respectivamente)

\begin{longtable}[]{@{}llll@{}}
\toprule\noalign{}
Operador & Comparación & Ejemplo & Resultado \\
\midrule\noalign{}
\endhead
\texttt{\textless{}} & Menor que & \texttt{5\ \textless{}\ 3} &
\texttt{FALSE} \\
\texttt{\textless{}=} & Menor o igual que & \texttt{5\ \textless{}=\ 3}
& \texttt{FALSE} \\
\texttt{\textgreater{}} & Mayor que & \texttt{5\ \textgreater{}\ 3} &
\texttt{TRUE} \\
\texttt{\textgreater{}=} & Mayor o igual que &
\texttt{5\ \textgreater{}=\ 3} & \texttt{TRUE} \\
\texttt{==} & Exactamente igual que & \texttt{5\ ==\ 3} &
\texttt{FALSE} \\
\texttt{!=} & No es igual que & \texttt{5\ !=\ 3} & \texttt{TRUE} \\
\bottomrule\noalign{}
\end{longtable}

Casi siempre cuando se hacen comparaciones siempre obtendremos TRUE o
FALSE.
\end{frame}

\begin{frame}[fragile]{Operadores lógicos}
\phantomsection\label{operadores-luxf3gicos}
Los operadores lógicos son usados para describir relaciones lógicas,
expresadas como verdadero o falso.

\begin{longtable}[]{@{}ll@{}}
\toprule\noalign{}
Operador & Comparación \\
\midrule\noalign{}
\endhead
\texttt{x\ \textbar{}\ y} & \texttt{x} ó \texttt{y} es verdadero \\
\texttt{x\ \&\ y} & \texttt{x} Y \texttt{y} son verdaderos \\
\texttt{!x} & \texttt{x} no es verdadero \\
\bottomrule\noalign{}
\end{longtable}

\begin{block}{Advertencia}
\phantomsection\label{advertencia}
\begin{itemize}
\item
  \texttt{\textbar{}} devuelve \texttt{TRUE} si alguno de los datos es
  \texttt{TRUE}
\item
  \texttt{\&} solo devuelve \texttt{TRUE} si ambos datos es
  \texttt{TRUE}
\item
  \texttt{\textbar{}} solo devuelve \texttt{FALSE} si ambos datos son
  \texttt{FALSE}
\item
  \texttt{\&} devuelve \texttt{FALSE} si alguno de los datos es
  \texttt{FALSE}
\end{itemize}
\end{block}
\end{frame}

\begin{frame}[fragile]{Operadores de asignación}
\phantomsection\label{operadores-de-asignaciuxf3n}
Los operadores de asignación son los más importantes, nos permiten
asignar datos a variables. Se recomienda utilizar \texttt{\textless{}-}
a la hora de asignar.

\begin{Shaded}
\begin{Highlighting}[]
\NormalTok{juan }\OtherTok{\textless{}{-}} \DecValTok{5}
\NormalTok{jose }\OtherTok{=}  \DecValTok{5}
\end{Highlighting}
\end{Shaded}

\begin{block}{Consideraciones:}
\phantomsection\label{consideraciones}
\begin{itemize}
\item
  Los nombres de las variables deben comenzar con una letra (a-z, A-Z).
\item
  No pueden comenzar con un número, contener espacios o caracteres
  especiales (\%, \$, \&, etc.), excepto el guion bajo (\_).
\item
  R es sensible a mayúsculas y minúsculas (por ejemplo, \texttt{edad} y
  \texttt{Edad} son variables diferentes).
\end{itemize}
\end{block}
\end{frame}

\section{Tipos de Datos y
Estructuras}\label{tipos-de-datos-y-estructuras}

\begin{frame}[fragile]{Tipos de datos}
\phantomsection\label{tipos-de-datos}
Los tipos de datos más común en R son:

\begin{longtable}[]{@{}lll@{}}
\toprule\noalign{}
Tipo & Ejemplo & Nombre en inglés \\
\midrule\noalign{}
\endhead
Entero & 1 & integer \\
Numérico & 1.3 & numeric \\
Cadena de texto & ``uno'' & character \\
Factor & uno & factor \\
Lógico & TRUE & logical \\
Perdido & \texttt{NA} & NA \\
Vacio & \texttt{NULL} & null \\
\bottomrule\noalign{}
\end{longtable}

Para consultar el tipo de datos, usaremos la función \texttt{class()}
\end{frame}

\begin{frame}[fragile]{Coerción}
\phantomsection\label{coerciuxf3n}
Podemos manipular los datos en \texttt{R}, o más precisamente, realizar
conversiones forzadas de un tipo a otro utilizando las funciones de la
familia \texttt{as}:
\texttt{as.interger(),\ as.numeric(),\ as.character(),\ as.factor(),\ as.logical(),\ as.null()}
. Todas estas funciones trabajan con datos y vectores.

\textbf{Ejemplos:}

\begin{Shaded}
\begin{Highlighting}[]
\FunctionTok{as.character}\NormalTok{(}\DecValTok{5}\NormalTok{)}
\end{Highlighting}
\end{Shaded}

\begin{verbatim}
## [1] "5"
\end{verbatim}

\begin{Shaded}
\begin{Highlighting}[]
\FunctionTok{as.logical}\NormalTok{(}\DecValTok{1}\NormalTok{)}
\end{Highlighting}
\end{Shaded}

\begin{verbatim}
## [1] TRUE
\end{verbatim}
\end{frame}

\begin{frame}[fragile]{Coerción}
\phantomsection\label{coerciuxf3n-1}
\begin{block}{Advertencia}
\phantomsection\label{advertencia-1}
Se debe utilizar esta funciones con cuidado, ya que pueden generar
resultados inesperados. Siempre se debe de evaluar antes de convertir un
tipo de dato a otro. Por ejemplo: si pasas un texto a numérico, el
resultado será \texttt{NA}. Si pasas un número a lógico, el resultado
será \texttt{TRUE} si el número es distinto de cero y \texttt{FALSE} si
es cero, pero si es mayor de 1, el resultado será \texttt{TRUE}.
\end{block}
\end{frame}

\begin{frame}[fragile]{Estructuras de datos}
\phantomsection\label{estructuras-de-datos}
\begin{block}{Vectores}
\phantomsection\label{vectores}
Un vector es una colección de uno o más datos del mismo tipos. Los
vectores siempre tienen tres propiedades:

\begin{enumerate}
\item
  \textbf{Tipo:} El tipo de dato que contiene el vector (numérico,
  carácter, lógico, etc.)
\item
  \textbf{Longitud:} El número de elementos en el vector.
\item
  \textbf{Atributos:} Metadatos adicionales que describen el vector
  (como nombres de elementos). Por ejemplo, un vector puede tener
  nombres para cada uno de sus elementos:
  \texttt{c(a\ =\ 1,\ b\ =\ 2,\ c\ =\ 3)}.
\end{enumerate}
\end{block}
\end{frame}

\begin{frame}[fragile]{Estructuras de datos}
\phantomsection\label{estructuras-de-datos-1}
\textbf{Vectores}

\begin{Shaded}
\begin{Highlighting}[]
\NormalTok{vec }\OtherTok{\textless{}{-}} \DecValTok{1} \CommentTok{\# Crear un vector}
\CommentTok{\# Vectores de más de un elemento usar c()}
\NormalTok{vector1 }\OtherTok{\textless{}{-}} \FunctionTok{c}\NormalTok{(}\DecValTok{1}\NormalTok{,}\DecValTok{3}\NormalTok{,}\DecValTok{4}\NormalTok{,}\DecValTok{5}\NormalTok{,}\DecValTok{6}\NormalTok{)}
\NormalTok{vector2 }\OtherTok{\textless{}{-}} \DecValTok{1}\SpecialCharTok{:}\DecValTok{10}

\NormalTok{vector2 }\SpecialCharTok{+} \DecValTok{2} \CommentTok{\# Vectorización}
\end{Highlighting}
\end{Shaded}

\begin{verbatim}
##  [1]  3  4  5  6  7  8  9 10 11 12
\end{verbatim}
\end{frame}

\begin{frame}[fragile]{Estructuras de datos}
\phantomsection\label{estructuras-de-datos-2}
\textbf{Data frames}

Un data frame es una colección de vectores de igual longitud, donde cada
vector representa una columna y cada fila representa una observación.
Los data frames son especialmente útiles para manejar conjuntos de datos
tabulares, como los que se encuentran comúnmente en análisis
estadísticos y científicos.

\begin{Shaded}
\begin{Highlighting}[]
\FunctionTok{head}\NormalTok{(iris, }\AttributeTok{n =} \FunctionTok{c}\NormalTok{(}\DecValTok{3}\NormalTok{,}\DecValTok{3}\NormalTok{))}
\end{Highlighting}
\end{Shaded}

\begin{verbatim}
##   Sepal.Length Sepal.Width Petal.Length
## 1          5.1         3.5          1.4
## 2          4.9         3.0          1.4
## 3          4.7         3.2          1.3
\end{verbatim}
\end{frame}

\begin{frame}[fragile]{Estructuras de datos}
\phantomsection\label{estructuras-de-datos-3}
\textbf{Creando un data frames}

\begin{Shaded}
\begin{Highlighting}[]
\CommentTok{\# Crear vectores}
\NormalTok{nombre }\OtherTok{\textless{}{-}} \FunctionTok{c}\NormalTok{(}\StringTok{"Ana"}\NormalTok{, }\StringTok{"Luis"}\NormalTok{, }\StringTok{"Marta"}\NormalTok{)}
\NormalTok{edad }\OtherTok{\textless{}{-}} \FunctionTok{c}\NormalTok{(}\DecValTok{23}\NormalTok{, }\DecValTok{30}\NormalTok{, }\DecValTok{25}\NormalTok{)}
\NormalTok{peso }\OtherTok{\textless{}{-}} \FunctionTok{c}\NormalTok{(}\FloatTok{55.5}\NormalTok{, }\FloatTok{70.2}\NormalTok{, }\FloatTok{60.3}\NormalTok{)}

\CommentTok{\# Crear data frame}
\NormalTok{estudiantes }\OtherTok{\textless{}{-}} \FunctionTok{data.frame}\NormalTok{(nombre, edad, peso)}
\end{Highlighting}
\end{Shaded}
\end{frame}

\begin{frame}[fragile]{Estructuras de datos}
\phantomsection\label{estructuras-de-datos-4}
\textbf{Acceder a los elementos de un data frame}

\begin{Shaded}
\begin{Highlighting}[]
\CommentTok{\# Acceder a una columna}
\NormalTok{estudiantes}\SpecialCharTok{$}\NormalTok{nombre}
\end{Highlighting}
\end{Shaded}

\begin{verbatim}
## [1] "Ana"   "Luis"  "Marta"
\end{verbatim}

\begin{Shaded}
\begin{Highlighting}[]
\CommentTok{\# Acceder a una fila}
\NormalTok{estudiantes[}\DecValTok{1}\NormalTok{, ]}
\end{Highlighting}
\end{Shaded}

\begin{verbatim}
##   nombre edad peso
## 1    Ana   23 55.5
\end{verbatim}
\end{frame}

\begin{frame}[fragile]{Estructuras de datos}
\phantomsection\label{estructuras-de-datos-5}
\textbf{Acceder a los elementos de un data frame}

\begin{Shaded}
\begin{Highlighting}[]
\CommentTok{\# Acceder a un elemento específico}
\NormalTok{estudiantes[}\DecValTok{1}\NormalTok{, }\DecValTok{2}\NormalTok{]}
\end{Highlighting}
\end{Shaded}

\begin{verbatim}
## [1] 23
\end{verbatim}

\begin{Shaded}
\begin{Highlighting}[]
\CommentTok{\# Acceder a múltiples filas y columnas}
\NormalTok{estudiantes[}\DecValTok{1}\SpecialCharTok{:}\DecValTok{2}\NormalTok{, }\FunctionTok{c}\NormalTok{(}\StringTok{"nombre"}\NormalTok{, }\StringTok{"edad"}\NormalTok{)]}
\end{Highlighting}
\end{Shaded}

\begin{verbatim}
##   nombre edad
## 1    Ana   23
## 2   Luis   30
\end{verbatim}
\end{frame}

\section{Ejercicios practicos}\label{ejercicios-practicos}

\begin{frame}[fragile]{Ejercicios}
\phantomsection\label{ejercicios}
\begin{enumerate}
\item
  Crea un vector llamado \texttt{edades} que contenga las edades de
  cinco personas. Ejemplo: 25, 30, 22, 28, 35. Luego, muestra el vector
  en la consola.
\item
  Utiliza operadores aritméticos para calcular la suma, resta,
  multiplicación y división de dos números. Asigna el resultado de cada
  operación a una variable diferente y muestra los resultados en la
  consola.
\item
  Crea un data frame llamado \texttt{estudiantes} con las siguientes
  columnas: \texttt{nombre} (carácter), \texttt{edad} (numérico) y
  \texttt{aprobado} (lógico). Llena el data frame con datos de cinco
  estudiantes.
\item
  Utiliza la función \texttt{class()} para verificar el tipo de datos de
  cada columna en el data frame \texttt{estudiantes}.
\end{enumerate}
\end{frame}

\begin{frame}[fragile]{Resultados}
\phantomsection\label{resultados}
\begin{Shaded}
\begin{Highlighting}[]
\CommentTok{\# Ejercicio 1}
\NormalTok{edades }\OtherTok{\textless{}{-}} \FunctionTok{c}\NormalTok{(}\DecValTok{25}\NormalTok{, }\DecValTok{30}\NormalTok{, }\DecValTok{22}\NormalTok{, }\DecValTok{28}\NormalTok{, }\DecValTok{35}\NormalTok{)}
\NormalTok{edades}
\end{Highlighting}
\end{Shaded}

\begin{verbatim}
## [1] 25 30 22 28 35
\end{verbatim}
\end{frame}

\begin{frame}[fragile]{Resultados}
\phantomsection\label{resultados-1}
\begin{Shaded}
\begin{Highlighting}[]
\CommentTok{\# Ejercicio 2}
\NormalTok{num1 }\OtherTok{\textless{}{-}} \DecValTok{10}
\NormalTok{num2 }\OtherTok{\textless{}{-}} \DecValTok{5}
\NormalTok{suma }\OtherTok{\textless{}{-}}\NormalTok{ num1 }\SpecialCharTok{+}\NormalTok{ num2}
\NormalTok{resta }\OtherTok{\textless{}{-}}\NormalTok{ num1 }\SpecialCharTok{{-}}\NormalTok{ num2}
\NormalTok{multiplicacion }\OtherTok{\textless{}{-}}\NormalTok{ num1 }\SpecialCharTok{*}\NormalTok{ num2}
\NormalTok{division }\OtherTok{\textless{}{-}}\NormalTok{ num1 }\SpecialCharTok{/}\NormalTok{ num2}
\NormalTok{suma; resta; division; multiplicacion}
\end{Highlighting}
\end{Shaded}

\begin{verbatim}
## [1] 15
\end{verbatim}

\begin{verbatim}
## [1] 5
\end{verbatim}

\begin{verbatim}
## [1] 2
\end{verbatim}

\begin{verbatim}
## [1] 50
\end{verbatim}
\end{frame}

\begin{frame}[fragile]{Resultados}
\phantomsection\label{resultados-2}
\begin{Shaded}
\begin{Highlighting}[]
\CommentTok{\# Ejercicio 3}
\NormalTok{nombre }\OtherTok{\textless{}{-}} \FunctionTok{c}\NormalTok{(}\StringTok{"Ana"}\NormalTok{, }\StringTok{"Luis"}\NormalTok{, }\StringTok{"Marta"}\NormalTok{, }\StringTok{"Carlos"}\NormalTok{, }\StringTok{"Sofia"}\NormalTok{)}
\NormalTok{edad }\OtherTok{\textless{}{-}} \FunctionTok{c}\NormalTok{(}\DecValTok{23}\NormalTok{, }\DecValTok{30}\NormalTok{, }\DecValTok{25}\NormalTok{, }\DecValTok{28}\NormalTok{, }\DecValTok{22}\NormalTok{)}
\NormalTok{aprobado }\OtherTok{\textless{}{-}} \FunctionTok{c}\NormalTok{(}\ConstantTok{TRUE}\NormalTok{, }\ConstantTok{FALSE}\NormalTok{, }\ConstantTok{TRUE}\NormalTok{, }\ConstantTok{TRUE}\NormalTok{, }\ConstantTok{FALSE}\NormalTok{)}
\NormalTok{estudiantes }\OtherTok{\textless{}{-}} \FunctionTok{data.frame}\NormalTok{(nombre, edad, aprobado)}
\NormalTok{estudiantes}
\end{Highlighting}
\end{Shaded}

\begin{verbatim}
##   nombre edad aprobado
## 1    Ana   23     TRUE
## 2   Luis   30    FALSE
## 3  Marta   25     TRUE
## 4 Carlos   28     TRUE
## 5  Sofia   22    FALSE
\end{verbatim}
\end{frame}

\begin{frame}[fragile]{Resultados}
\phantomsection\label{resultados-3}
\begin{Shaded}
\begin{Highlighting}[]
\CommentTok{\# Ejercicio 4}
\FunctionTok{class}\NormalTok{(estudiantes}\SpecialCharTok{$}\NormalTok{nombre)}
\end{Highlighting}
\end{Shaded}

\begin{verbatim}
## [1] "character"
\end{verbatim}

\begin{Shaded}
\begin{Highlighting}[]
\CommentTok{\# Ejercicio 4}
\FunctionTok{class}\NormalTok{(estudiantes}\SpecialCharTok{$}\NormalTok{aprobado)}
\end{Highlighting}
\end{Shaded}

\begin{verbatim}
## [1] "logical"
\end{verbatim}
\end{frame}

\section{Funciones y paquetes}\label{funciones-y-paquetes}

\begin{frame}[fragile]{Funciones}
\phantomsection\label{funciones}
Las funciones son bloques de código reutilizables que realizan tareas
específicas. En R, las funciones se definen utilizando la palabra clave
\texttt{function()}. Las funciones pueden aceptar argumentos (entradas)
y devolver valores (salidas).

\begin{Shaded}
\begin{Highlighting}[]
\CommentTok{\# Estructura básica de una función}
\NormalTok{mi\_funcion }\OtherTok{\textless{}{-}} \ControlFlowTok{function}\NormalTok{(arg1, arg2) \{}
  \CommentTok{\# Código que realiza una tarea}
\NormalTok{  resultado }\OtherTok{\textless{}{-}}\NormalTok{ arg1 }\SpecialCharTok{+}\NormalTok{ arg2}
  \FunctionTok{return}\NormalTok{(resultado)}
\NormalTok{\}}
\end{Highlighting}
\end{Shaded}

Si queremos acceder a la documentación de una función en especial,
podemos ejecutar \texttt{?mean} o \texttt{help(mean)}
\end{frame}

\begin{frame}[fragile]{Paquetes}
\phantomsection\label{paquetes}
Los paquetes son colecciones de funciones, datos y documentación que
extienden las capacidades básicas de R. Existen miles de paquetes
disponibles para diversas tareas, desde análisis estadísticos hasta
visualización de datos.

\textbf{Instalación y carga de paquetes}

\begin{Shaded}
\begin{Highlighting}[]
\CommentTok{\# Para instalar un paquete, usamos la función}
\FunctionTok{install.packages}\NormalTok{(}\StringTok{"tidyverse"}\NormalTok{) }

\CommentTok{\# Una vez instalado, debemos cargar el paquete }
\CommentTok{\# en nuestra sesión de R utilizando:}
\FunctionTok{library}\NormalTok{(tidyverse)}
\end{Highlighting}
\end{Shaded}
\end{frame}

\begin{frame}[fragile]{Funciones más usadas en dplyr}
\phantomsection\label{funciones-muxe1s-usadas-en-dplyr}
\begin{longtable}[]{@{}ll@{}}
\toprule\noalign{}
Función & Descripción \\
\midrule\noalign{}
\endhead
\texttt{filter()} & Filtra filas según condiciones específicas. \\
\texttt{select()} & Selecciona columnas específicas de un data frame. \\
\texttt{group\_by()} & Agrupa datos según una o más columnas. \\
\texttt{mutate()} & Crea nuevas columnas o modifica existentes. \\
\texttt{arrange()} & Ordena filas según una o más columnas. \\
\texttt{summarise()} & Resume múltiples valores en un solo valor. \\
\bottomrule\noalign{}
\end{longtable}
\end{frame}

\begin{frame}[fragile]{Ejemplos}
\phantomsection\label{ejemplos}
\begin{Shaded}
\begin{Highlighting}[]
\FunctionTok{library}\NormalTok{(tidyverse)}
\FunctionTok{filter}\NormalTok{(estudiantes, edad }\SpecialCharTok{\textgreater{}} \DecValTok{25}\NormalTok{)}
\end{Highlighting}
\end{Shaded}

\begin{verbatim}
##   nombre edad aprobado
## 1   Luis   30    FALSE
## 2 Carlos   28     TRUE
\end{verbatim}

\begin{Shaded}
\begin{Highlighting}[]
\FunctionTok{select}\NormalTok{(estudiantes, nombre, edad)}
\end{Highlighting}
\end{Shaded}

\begin{verbatim}
##   nombre edad
## 1    Ana   23
## 2   Luis   30
## 3  Marta   25
## 4 Carlos   28
## 5  Sofia   22
\end{verbatim}
\end{frame}

\begin{frame}[fragile]{Ejemplos}
\phantomsection\label{ejemplos-1}
\begin{Shaded}
\begin{Highlighting}[]
\FunctionTok{summarise}\NormalTok{(}
  \FunctionTok{group\_by}\NormalTok{(estudiantes, aprobado),}
  \AttributeTok{edad\_media =} \FunctionTok{mean}\NormalTok{(edad), }\AttributeTok{edad\_max =} \FunctionTok{max}\NormalTok{(edad)}
\NormalTok{  )}
\end{Highlighting}
\end{Shaded}

\begin{verbatim}
## # A tibble: 2 x 3
##   aprobado edad_media edad_max
##   <lgl>         <dbl>    <dbl>
## 1 FALSE          26         30
## 2 TRUE           25.3       28
\end{verbatim}
\end{frame}

\begin{frame}[fragile]{Pipe}
\phantomsection\label{pipe}
El operador pipe (\texttt{\%\textgreater{}\%} o
\texttt{\textbar{}\textgreater{}}) es una herramienta poderosa en R que
permite encadenar múltiples operaciones de manera legible y concisa.
Facilita la lectura del código al expresar una secuencia de
transformaciones de datos.

\begin{Shaded}
\begin{Highlighting}[]
\CommentTok{\# Ejemplo usando el operador pipe}
\NormalTok{estudiantes }\SpecialCharTok{\%\textgreater{}\%}
  \FunctionTok{filter}\NormalTok{(edad }\SpecialCharTok{\textgreater{}} \DecValTok{25}\NormalTok{) }\SpecialCharTok{\%\textgreater{}\%}
  \FunctionTok{select}\NormalTok{(nombre, edad) }\SpecialCharTok{\%\textgreater{}\%}
  \FunctionTok{arrange}\NormalTok{(}\FunctionTok{desc}\NormalTok{(edad))}
\end{Highlighting}
\end{Shaded}

\begin{verbatim}
##   nombre edad
## 1   Luis   30
## 2 Carlos   28
\end{verbatim}
\end{frame}

\section{Importación de datos}\label{importaciuxf3n-de-datos}

\begin{frame}[fragile]{Importación de datos}
\phantomsection\label{importaciuxf3n-de-datos-1}
Podemos cargar datos de diferentes formatos, como CSV, Excel, entre
otros. Vamos a cargar la base de datos de la Encuesta Nacional de Fuerza
de Trabajo (ENFCT) que se encuentra en formato excel.

Con el paquete \texttt{readxl} podemos leer archivos de Excel (.xlsx,
.xls).

\begin{Shaded}
\begin{Highlighting}[]
\FunctionTok{library}\NormalTok{(readxl)}
\NormalTok{enfct\_2019 }\OtherTok{\textless{}{-}} \FunctionTok{read\_excel}\NormalTok{(}\StringTok{"enfct\_2019.xlsx"}\NormalTok{)}
\end{Highlighting}
\end{Shaded}
\end{frame}

\begin{frame}[fragile]{Evaluación de datos}
\phantomsection\label{evaluaciuxf3n-de-datos}
Una vez que hemos importado los datos, es importante evaluar su
estructura y contenido para entender qué tipo de análisis podemos
realizar. Algunas funciones útiles para esta tarea son:

\begin{Shaded}
\begin{Highlighting}[]
\CommentTok{\# Ver las primeras filas del data frame}
\FunctionTok{head}\NormalTok{(enfct\_2019)}
\CommentTok{\# La más recomendables es glimpse() del paquete dplyr}
\FunctionTok{glimpse}\NormalTok{(enfct\_2019)}
\end{Highlighting}
\end{Shaded}
\end{frame}

\section{Manipulación de datos con
dplyr}\label{manipulaciuxf3n-de-datos-con-dplyr}

\begin{frame}[fragile]{Manipulación de datos con dplyr}
\phantomsection\label{manipulaciuxf3n-de-datos-con-dplyr-1}
El paquete \texttt{dplyr} es parte del ecosistema \texttt{tidyverse} y
proporciona una gramática coherente para manipular datos. A
continuación, se presentan algunas de las funciones más comunes y útiles
de \texttt{dplyr} para transformar y analizar datos.

\begin{block}{Tips opcional}
\phantomsection\label{tips-opcional}
Se recomienda limpiar los nombres de las variables, para evitar
problemas al momento de manipular los datos. Para esto, usaremos la
función \texttt{clean\_names()} del paquete \texttt{janitor}.
\end{block}
\end{frame}

\begin{frame}[fragile]{Manipulación de datos con dplyr}
\phantomsection\label{manipulaciuxf3n-de-datos-con-dplyr-2}
\begin{Shaded}
\begin{Highlighting}[]
\FunctionTok{library}\NormalTok{(janitor)}
\NormalTok{enfct\_2019 }\OtherTok{\textless{}{-}} \FunctionTok{clean\_names}\NormalTok{(enfct\_2019)}
\end{Highlighting}
\end{Shaded}

\begin{Shaded}
\begin{Highlighting}[]
\NormalTok{enfct\_2019  }\SpecialCharTok{\%\textgreater{}\%} 
  \FunctionTok{select}\NormalTok{(}\DecValTok{1}\SpecialCharTok{:}\DecValTok{3}\NormalTok{) }\SpecialCharTok{\%\textgreater{}\%} 
  \FunctionTok{tail}\NormalTok{(}\DecValTok{3}\NormalTok{) }\SpecialCharTok{\%\textgreater{}\%} 
  \FunctionTok{glimpse}\NormalTok{()}
\end{Highlighting}
\end{Shaded}

\begin{verbatim}
## Rows: 3
## Columns: 3
## $ ano              <dbl> 2019, 2019, 2019
## $ id_provincia     <dbl> 12, 12, 12
## $ factor_expansion <dbl> 173.7651, 173.7651, 173.7651
\end{verbatim}
\end{frame}

\begin{frame}[fragile]{Manipulación de datos con dplyr}
\phantomsection\label{manipulaciuxf3n-de-datos-con-dplyr-3}
\textbf{Primer paso:}

Definimos las variables que nos interesa y evaluamos su contenido. En
este ejercicio vamos a selecionar las variables \texttt{edad},
\texttt{sexo}, \texttt{des\_provincia},
\texttt{sueldo\_bruto\_ap\_monto}, \texttt{estado\_civil} y
\texttt{nivel\_ultimo\_ano\_aprobado}.

\begin{Shaded}
\begin{Highlighting}[]
\NormalTok{enfct\_2019 }\OtherTok{\textless{}{-}}\NormalTok{ enfct\_2019 }\SpecialCharTok{\%\textgreater{}\%} 
  \FunctionTok{select}\NormalTok{(edad, sexo, des\_provincia, }
\NormalTok{         sueldo\_bruto\_ap\_monto, estado\_civil, }
\NormalTok{         nivel\_ultimo\_ano\_aprobado)}
\end{Highlighting}
\end{Shaded}
\end{frame}

\begin{frame}[fragile]{Manipulación de datos con dplyr}
\phantomsection\label{manipulaciuxf3n-de-datos-con-dplyr-4}
Evaluamos el contenido de las variables e identificamos los cambios
necesarios para nuestros datos.

\begin{Shaded}
\begin{Highlighting}[]
\NormalTok{enfct\_2019 }\SpecialCharTok{\%\textgreater{}\%}
  \FunctionTok{tail}\NormalTok{(}\DecValTok{3}\NormalTok{) }\SpecialCharTok{\%\textgreater{}\%}
  \FunctionTok{glimpse}\NormalTok{()}
\end{Highlighting}
\end{Shaded}

\begin{verbatim}
## Rows: 3
## Columns: 6
## $ edad                      <dbl> 26, 4, 32
## $ sexo                      <chr> "F", "F", "M"
## $ des_provincia             <chr> "LA ROMANA", "LA ROMANA", "LA ROMANA"
## $ sueldo_bruto_ap_monto     <dbl> 2500, NA, 2000
## $ estado_civil              <chr> "Union libre", NA, "Separado(a)"
## $ nivel_ultimo_ano_aprobado <chr> "Secundario", "Ninguno", "Primario"
\end{verbatim}
\end{frame}

\begin{frame}[fragile]{Manipulación de datos con dplyr}
\phantomsection\label{manipulaciuxf3n-de-datos-con-dplyr-5}
Con count, podemos identificar varios detalles: los valores dentro de la
variables y la frecuencia de cada valor. Una segunda opción es la
función \texttt{distinct}, que nos muestra los valores únicos dentro de
una variable.

\begin{Shaded}
\begin{Highlighting}[]
\NormalTok{enfct\_2019 }\SpecialCharTok{\%\textgreater{}\%}
  \FunctionTok{count}\NormalTok{(sexo)}
\end{Highlighting}
\end{Shaded}

\begin{verbatim}
## # A tibble: 2 x 2
##   sexo      n
##   <chr> <int>
## 1 F     41872
## 2 M     41159
\end{verbatim}
\end{frame}

\begin{frame}[fragile]{Manipulación de datos con dplyr}
\phantomsection\label{manipulaciuxf3n-de-datos-con-dplyr-6}
\begin{Shaded}
\begin{Highlighting}[]
\NormalTok{enfct\_2019 }\SpecialCharTok{\%\textgreater{}\%}
  \FunctionTok{distinct}\NormalTok{(estado\_civil)}
\end{Highlighting}
\end{Shaded}

\begin{verbatim}
## # A tibble: 7 x 1
##   estado_civil 
##   <chr>        
## 1 Separado(a)  
## 2 <NA>         
## 3 Casado(a)    
## 4 Soltero(a)   
## 5 Union libre  
## 6 Viudo(a)     
## 7 Divorciado(a)
\end{verbatim}
\end{frame}

\begin{frame}[fragile]{Manipulación de datos con dplyr}
\phantomsection\label{manipulaciuxf3n-de-datos-con-dplyr-7}
Vamos a realizar algunos cambios en las variables que seleccionamos.
Primero, vamos a cambiar los valores de la variable \texttt{sexo}, que
actualmente tiene los valores \texttt{F} y \texttt{M}, por
\texttt{Femenino} y \texttt{Masculino}. Para esto, usaremos la función
\texttt{mutate()} junto con recode.

\begin{Shaded}
\begin{Highlighting}[]
\NormalTok{enfct\_2019 }\OtherTok{\textless{}{-}}\NormalTok{ enfct\_2019 }\SpecialCharTok{\%\textgreater{}\%}
  \FunctionTok{mutate}\NormalTok{(}\AttributeTok{sexo =} \FunctionTok{recode}\NormalTok{(sexo, }
                       \StringTok{"F"} \OtherTok{=} \DecValTok{1}\NormalTok{, }
                       \StringTok{"M"} \OtherTok{=} \DecValTok{0}\NormalTok{))}
\NormalTok{enfct\_2019 }\SpecialCharTok{\%\textgreater{}\%} \FunctionTok{distinct}\NormalTok{(sexo)}
\end{Highlighting}
\end{Shaded}

\begin{verbatim}
## # A tibble: 2 x 1
##    sexo
##   <dbl>
## 1     1
## 2     0
\end{verbatim}
\end{frame}

\begin{frame}[fragile]{Manipulación de datos con dplyr}
\phantomsection\label{manipulaciuxf3n-de-datos-con-dplyr-8}
Vamos a cambiar los valores de la variable \texttt{estado\_civil}, que
actualmente tiene los valores Separado(a), Casado(a), Soltero(a), Union
libre, Viudo(a), Divorciado(a), por \texttt{Casado} y \texttt{Soltero}.
Para esto, usaremos la función \texttt{mutate()} junto con
\texttt{case\_when()}.

\begin{Shaded}
\begin{Highlighting}[]
\NormalTok{enfct\_2019 }\OtherTok{\textless{}{-}}\NormalTok{ enfct\_2019 }\SpecialCharTok{\%\textgreater{}\%} 
  \FunctionTok{mutate}\NormalTok{(}\AttributeTok{estado\_civil =} \FunctionTok{case\_when}\NormalTok{(}
\NormalTok{    estado\_civil }\SpecialCharTok{\%in\%} \FunctionTok{c}\NormalTok{(}
      \StringTok{"Casado(a)"}\NormalTok{, }\StringTok{"Union libre"}\NormalTok{) }\SpecialCharTok{\textasciitilde{}} \StringTok{"Casado"}\NormalTok{,}
\NormalTok{    estado\_civil }\SpecialCharTok{\%in\%} \FunctionTok{c}\NormalTok{(}
      \StringTok{"Soltero(a)"}\NormalTok{, }\StringTok{"Viudo(a)"}\NormalTok{, }\StringTok{"Divorciado(a)"}\NormalTok{, }
      \StringTok{"Separado(a)"}\NormalTok{) }\SpecialCharTok{\textasciitilde{}} \StringTok{"Soltero"}
\NormalTok{  ))}
\end{Highlighting}
\end{Shaded}
\end{frame}

\begin{frame}[fragile]{Manipulación de datos con dplyr}
\phantomsection\label{manipulaciuxf3n-de-datos-con-dplyr-9}
\begin{Shaded}
\begin{Highlighting}[]
\NormalTok{enfct\_2019 }\SpecialCharTok{\%\textgreater{}\%} \FunctionTok{distinct}\NormalTok{(estado\_civil)}
\end{Highlighting}
\end{Shaded}

\begin{verbatim}
## # A tibble: 3 x 1
##   estado_civil
##   <chr>       
## 1 Soltero     
## 2 <NA>        
## 3 Casado
\end{verbatim}
\end{frame}

\section{Ejercicios prácticos}\label{ejercicios-pruxe1cticos}

\begin{frame}[fragile]{Ejercicios}
\phantomsection\label{ejercicios-1}
\begin{enumerate}
\item
  Carga el conjunto de datos \texttt{mtcars} que viene incluido en R.
  Utiliza la función \texttt{head()} para ver las primeras filas del
  conjunto de datos.
\item
  Selecciona las columnas \texttt{mpg}, \texttt{cyl}, y \texttt{hp} del
  conjunto de datos \texttt{mtcars} y crea un nuevo data frame llamado
  \texttt{mtcars\_subset}.
\item
  Filtra las filas del data frame \texttt{mtcars\_subset} para incluir
  solo los autos con más de 6 cilindros
  (\texttt{cyl\ \textgreater{}\ 6}). Guarda el resultado en un nuevo
  data frame llamado \texttt{mtcars\_filtered}.
\item
  Crea una nueva columna en el data frame \texttt{mtcars\_filtered}
  llamada \texttt{power\_to\_weight} que sea el resultado de dividir la
  potencia (\texttt{hp}) por el peso (\texttt{wt}). Utiliza la función
  \texttt{mutate()} para esto.
\end{enumerate}
\end{frame}

\section{Resultados}\label{resultados-4}

\begin{frame}[fragile]{Resultados}
\phantomsection\label{resultados-5}
\begin{Shaded}
\begin{Highlighting}[]
\FunctionTok{data}\NormalTok{(mtcars)}
\CommentTok{\# Ejercicio 1}
\FunctionTok{head}\NormalTok{(mtcars, }\DecValTok{6}\NormalTok{)}
\end{Highlighting}
\end{Shaded}

\begin{verbatim}
##                    mpg cyl disp  hp drat    wt  qsec vs am gear carb
## Mazda RX4         21.0   6  160 110 3.90 2.620 16.46  0  1    4    4
## Mazda RX4 Wag     21.0   6  160 110 3.90 2.875 17.02  0  1    4    4
## Datsun 710        22.8   4  108  93 3.85 2.320 18.61  1  1    4    1
## Hornet 4 Drive    21.4   6  258 110 3.08 3.215 19.44  1  0    3    1
## Hornet Sportabout 18.7   8  360 175 3.15 3.440 17.02  0  0    3    2
## Valiant           18.1   6  225 105 2.76 3.460 20.22  1  0    3    1
\end{verbatim}
\end{frame}

\begin{frame}[fragile]{Resultados}
\phantomsection\label{resultados-6}
\begin{Shaded}
\begin{Highlighting}[]
\CommentTok{\# Ejercicio 2}
\NormalTok{mtcars }\SpecialCharTok{\%\textgreater{}\%}
  \FunctionTok{select}\NormalTok{(mpg, cyl, hp) }\SpecialCharTok{\%\textgreater{}\%} 
  \FunctionTok{head}\NormalTok{(}\DecValTok{6}\NormalTok{)}
\end{Highlighting}
\end{Shaded}

\begin{verbatim}
##                    mpg cyl  hp
## Mazda RX4         21.0   6 110
## Mazda RX4 Wag     21.0   6 110
## Datsun 710        22.8   4  93
## Hornet 4 Drive    21.4   6 110
## Hornet Sportabout 18.7   8 175
## Valiant           18.1   6 105
\end{verbatim}
\end{frame}

\begin{frame}[fragile]{Resultados}
\phantomsection\label{resultados-7}
\begin{Shaded}
\begin{Highlighting}[]
\CommentTok{\# Ejercicio 3}
\NormalTok{mtcars }\SpecialCharTok{\%\textgreater{}\%}
  \FunctionTok{filter}\NormalTok{(cyl }\SpecialCharTok{\textgreater{}} \DecValTok{6}\NormalTok{) }\SpecialCharTok{\%\textgreater{}\%} 
  \FunctionTok{head}\NormalTok{(}\DecValTok{6}\NormalTok{)}
\end{Highlighting}
\end{Shaded}

\begin{verbatim}
##                     mpg cyl  disp  hp drat   wt  qsec vs am gear carb
## Hornet Sportabout  18.7   8 360.0 175 3.15 3.44 17.02  0  0    3    2
## Duster 360         14.3   8 360.0 245 3.21 3.57 15.84  0  0    3    4
## Merc 450SE         16.4   8 275.8 180 3.07 4.07 17.40  0  0    3    3
## Merc 450SL         17.3   8 275.8 180 3.07 3.73 17.60  0  0    3    3
## Merc 450SLC        15.2   8 275.8 180 3.07 3.78 18.00  0  0    3    3
## Cadillac Fleetwood 10.4   8 472.0 205 2.93 5.25 17.98  0  0    3    4
\end{verbatim}
\end{frame}

\begin{frame}[fragile]{Resultados}
\phantomsection\label{resultados-8}
\begin{Shaded}
\begin{Highlighting}[]
\CommentTok{\# Ejercicio 4}
\NormalTok{mtcars }\SpecialCharTok{\%\textgreater{}\%}
  \FunctionTok{mutate}\NormalTok{(}
    \AttributeTok{power\_to\_weight =}\NormalTok{ hp }\SpecialCharTok{/}\NormalTok{ wt,}
    \AttributeTok{.before =}\NormalTok{ mpg}
\NormalTok{    ) }\SpecialCharTok{\%\textgreater{}\%} 
  \FunctionTok{head}\NormalTok{(}\DecValTok{6}\NormalTok{)}
\end{Highlighting}
\end{Shaded}

\begin{verbatim}
##                   power_to_weight  mpg cyl disp  hp drat    wt  qsec vs am gear
## Mazda RX4                41.98473 21.0   6  160 110 3.90 2.620 16.46  0  1    4
## Mazda RX4 Wag            38.26087 21.0   6  160 110 3.90 2.875 17.02  0  1    4
## Datsun 710               40.08621 22.8   4  108  93 3.85 2.320 18.61  1  1    4
## Hornet 4 Drive           34.21462 21.4   6  258 110 3.08 3.215 19.44  1  0    3
## Hornet Sportabout        50.87209 18.7   8  360 175 3.15 3.440 17.02  0  0    3
## Valiant                  30.34682 18.1   6  225 105 2.76 3.460 20.22  1  0    3
##                   carb
## Mazda RX4            4
## Mazda RX4 Wag        4
## Datsun 710           1
## Hornet 4 Drive       1
## Hornet Sportabout    2
## Valiant              1
\end{verbatim}
\end{frame}

\section{Exportación de datos}\label{exportaciuxf3n-de-datos}

\begin{frame}{Exportación de datos}
\phantomsection\label{exportaciuxf3n-de-datos-1}
Una vez que hemos manipulado y analizado nuestros datos, es importante
guardar los resultados para su uso futuro o para compartirlos con otros.
R ofrece varias funciones para exportar datos a diferentes formatos,
como CSV, Excel y RDS.
\end{frame}

\begin{frame}[fragile]{Exportar a excel con openxlsx}
\phantomsection\label{exportar-a-excel-con-openxlsx}
El paquete \texttt{openxlsx} permite exportar data frames a archivos de
Excel (.xlsx) de manera sencilla y eficiente.

\begin{Shaded}
\begin{Highlighting}[]
\CommentTok{\# Pasos para exportar un data frame a Excel}
\FunctionTok{library}\NormalTok{(openxlsx)}
\FunctionTok{write.xlsx}\NormalTok{(enfct\_2019, }\AttributeTok{file =} \StringTok{"enfct\_2019\_new.xlsx"}\NormalTok{)}
\end{Highlighting}
\end{Shaded}
\end{frame}

\section{Estadística descriptiva}\label{estaduxedstica-descriptiva}

\begin{frame}{Medidas de tendencia central}
\phantomsection\label{medidas-de-tendencia-central}
Las medidas de tendencia central son estadísticas que describen el
centro o la ubicación típica de un conjunto de datos. Las tres medidas
más comunes son la media, la mediana y la moda.

\[
\text{Media} = \frac{\sum_{i=1}^{n} x_i}{n}
\]

\[
\text{Mediana} =
\begin{cases}
\frac{x_{(n/2)} + x_{(n/2 + 1)}}{2}, & \text{si } n \text{ es par} \\
x_{((n + 1)/2)}, & \text{si } n \text{ es impar}
\end{cases}
\]
\end{frame}

\begin{frame}[fragile]{Medidas de tendencia central}
\phantomsection\label{medidas-de-tendencia-central-1}
\begin{longtable}[]{@{}
  >{\raggedright\arraybackslash}p{(\columnwidth - 4\tabcolsep) * \real{0.3333}}
  >{\raggedright\arraybackslash}p{(\columnwidth - 4\tabcolsep) * \real{0.3333}}
  >{\raggedright\arraybackslash}p{(\columnwidth - 4\tabcolsep) * \real{0.3333}}@{}}
\toprule\noalign{}
\begin{minipage}[b]{\linewidth}\raggedright
Medida
\end{minipage} & \begin{minipage}[b]{\linewidth}\raggedright
Descripción
\end{minipage} & \begin{minipage}[b]{\linewidth}\raggedright
Función en R
\end{minipage} \\
\midrule\noalign{}
\endhead
Media & El promedio aritmético de los datos. & \texttt{mean()} \\
Mediana & El valor central cuando los datos están ordenados. &
\texttt{median()} \\
Moda & El valor que aparece con mayor frecuencia. & No hay función
nativa, pero se puede calcular con \texttt{table()} y
\texttt{which.max()} \\
\bottomrule\noalign{}
\end{longtable}
\end{frame}

\begin{frame}[fragile]{Medidas de tendencia central}
\phantomsection\label{medidas-de-tendencia-central-2}
\begin{Shaded}
\begin{Highlighting}[]
\NormalTok{enfct\_2019 }\SpecialCharTok{\%\textgreater{}\%} 
  \FunctionTok{summarise}\NormalTok{(}
    \CommentTok{\# na.rm = TRUE para ignorar los valores perdidos}
    \AttributeTok{media =} \FunctionTok{mean}\NormalTok{(edad, }\AttributeTok{na.rm =} \ConstantTok{TRUE}\NormalTok{), }
    \AttributeTok{mediana =} \FunctionTok{median}\NormalTok{(edad, }\AttributeTok{na.rm =} \ConstantTok{TRUE}\NormalTok{)}
\NormalTok{  )}
\end{Highlighting}
\end{Shaded}

\begin{verbatim}
## # A tibble: 1 x 2
##   media mediana
##   <dbl>   <dbl>
## 1  31.3      28
\end{verbatim}
\end{frame}

\begin{frame}{Medidas de dispersión}
\phantomsection\label{medidas-de-dispersiuxf3n}
Las medidas de dispersión describen la variabilidad o dispersión de un
conjunto de datos. Las dos medidas más comunes son la desviación
estándar y la varianza.

\[
\text{\} = \frac{\sum_{i=1}^{n} (x_i - \bar{x})^2}{n - 1}
\]

\[
\text{Desviación Estándar} = \sqrt{\text{Varianza}}
\]
\end{frame}

\begin{frame}[fragile]{Medidas de dispersión}
\phantomsection\label{medidas-de-dispersiuxf3n-1}
\begin{longtable}[]{@{}
  >{\raggedright\arraybackslash}p{(\columnwidth - 4\tabcolsep) * \real{0.3333}}
  >{\raggedright\arraybackslash}p{(\columnwidth - 4\tabcolsep) * \real{0.3333}}
  >{\raggedright\arraybackslash}p{(\columnwidth - 4\tabcolsep) * \real{0.3333}}@{}}
\toprule\noalign{}
\begin{minipage}[b]{\linewidth}\raggedright
Medida
\end{minipage} & \begin{minipage}[b]{\linewidth}\raggedright
Descripción
\end{minipage} & \begin{minipage}[b]{\linewidth}\raggedright
Función en R
\end{minipage} \\
\midrule\noalign{}
\endhead
Varianza & La media de las desviaciones al cuadrado respecto a la media.
& \texttt{var()} \\
Desviación Estándar & La raíz cuadrada de la varianza, que mide la
dispersión en las mismas unidades que los datos. & \texttt{sd()} \\
\bottomrule\noalign{}
\end{longtable}
\end{frame}

\begin{frame}[fragile]{Medidas de dispersión}
\phantomsection\label{medidas-de-dispersiuxf3n-2}
\begin{Shaded}
\begin{Highlighting}[]
\NormalTok{enfct\_2019 }\SpecialCharTok{\%\textgreater{}\%} 
  \FunctionTok{summarise}\NormalTok{(}
    \AttributeTok{varianza =} \FunctionTok{var}\NormalTok{(edad, }\AttributeTok{na.rm =} \ConstantTok{TRUE}\NormalTok{), }
    \AttributeTok{desviacion\_estandar =} \FunctionTok{sd}\NormalTok{(edad, }\AttributeTok{na.rm =} \ConstantTok{TRUE}\NormalTok{)}
\NormalTok{  )}
\end{Highlighting}
\end{Shaded}

\begin{verbatim}
## # A tibble: 1 x 2
##   varianza desviacion_estandar
##      <dbl>               <dbl>
## 1     466.                21.6
\end{verbatim}
\end{frame}

\begin{frame}{Medidas de posición}
\phantomsection\label{medidas-de-posiciuxf3n}
Las medidas de posición describen la ubicación relativa de un valor
dentro de un conjunto de datos. Las medidas más comunes son los
cuartiles, percentiles y cuantiles.

\textbf{Percentil:} El percentil \(P_k\) es el valor que divide el
conjunto de datos en (k) partes iguales. Por ejemplo, el percentil 25
(P25) es el valor por debajo del cual se encuentra el 25\% de los datos.

\[
P_k = \text{Valor que divide el conjunto de datos en } k \text{ partes iguales}
\]
\end{frame}

\begin{frame}{Medidas de posición}
\phantomsection\label{medidas-de-posiciuxf3n-1}
\textbf{Cuartiles:} Los cuartiles \(Q_1\), \(Q_2\) y \(Q_3\) son valores
que dividen el conjunto de datos en cuatro partes iguales. \(Q_1\) es el
percentil 25, \(Q_2\) es la mediana (percentil 50) y \(Q_3\) es el
percentil 75.

\[
Q_1, Q_2, Q_3 = \text{Valores que dividen el conjunto de datos en cuatro partes iguales}
\]

\textbf{Cuantiles:} Los cuantiles \(q_p\) son valores que dividen el
conjunto de datos en \(p\) partes iguales. Por ejemplo, el cuantil 0.25
(q0.25) es el valor por debajo del cual se encuentra el 25\% de los
datos.

\[
q_p = \text{Valor que divide el conjunto de datos en } p \text{ partes iguales}
\]
\end{frame}

\begin{frame}[fragile]{Resumiendo nuestras variables}
\phantomsection\label{resumiendo-nuestras-variables}
\begin{Shaded}
\begin{Highlighting}[]
\NormalTok{enfct\_2019 }\SpecialCharTok{\%\textgreater{}\%} 
  \FunctionTok{summarise}\NormalTok{(}
    \AttributeTok{q1 =} \FunctionTok{quantile}\NormalTok{(edad, }\FloatTok{0.25}\NormalTok{, }\AttributeTok{na.rm =} \ConstantTok{TRUE}\NormalTok{),}
    \AttributeTok{q2 =} \FunctionTok{quantile}\NormalTok{(edad, }\FloatTok{0.50}\NormalTok{, }\AttributeTok{na.rm =} \ConstantTok{TRUE}\NormalTok{),}
    \AttributeTok{q3 =} \FunctionTok{quantile}\NormalTok{(edad, }\FloatTok{0.75}\NormalTok{, }\AttributeTok{na.rm =} \ConstantTok{TRUE}\NormalTok{)}
\NormalTok{  )}
\end{Highlighting}
\end{Shaded}

\begin{verbatim}
## # A tibble: 1 x 3
##      q1    q2    q3
##   <dbl> <dbl> <dbl>
## 1    13    28    47
\end{verbatim}
\end{frame}

\section{Gráficos en R con ggplot2}\label{gruxe1ficos-en-r-con-ggplot2}

\begin{frame}[fragile]{Gráficos de R base}
\phantomsection\label{gruxe1ficos-de-r-base}
R tiene funciones nativas para crear gráficos básicos, como
\texttt{plot()}, \texttt{hist()}, \texttt{boxplot()}, entre otras. Estas
funciones son fáciles de usar y permiten crear gráficos rápidamente. Su
desventaja principal es que la personalización y la creación de gráficos
complejos pueden ser limitadas.
\end{frame}

\begin{frame}[fragile]{Dispersión en R base}
\phantomsection\label{dispersiuxf3n-en-r-base}
\begin{Shaded}
\begin{Highlighting}[]
\CommentTok{\# Gráfico de dispersión}
\FunctionTok{plot}\NormalTok{(enfct\_2019}\SpecialCharTok{$}\NormalTok{edad, enfct\_2019}\SpecialCharTok{$}\NormalTok{sueldo\_bruto\_ap\_monto)}
\end{Highlighting}
\end{Shaded}

\includegraphics{presentacion-modulo1_files/figure-beamer/unnamed-chunk-38-1.pdf}
\end{frame}

\begin{frame}[fragile]{Histograma en R base}
\phantomsection\label{histograma-en-r-base}
\begin{Shaded}
\begin{Highlighting}[]
\FunctionTok{hist}\NormalTok{(enfct\_2019}\SpecialCharTok{$}\NormalTok{edad, }\AttributeTok{main =} \StringTok{"Histograma de Edad"}\NormalTok{)}
\end{Highlighting}
\end{Shaded}

\includegraphics{presentacion-modulo1_files/figure-beamer/unnamed-chunk-39-1.pdf}
\end{frame}

\begin{frame}[fragile]{Gráficos con ggplot2}
\phantomsection\label{gruxe1ficos-con-ggplot2}
Los gráficos con \texttt{ggplot2} se basan en la ``gramática de
gráficos'', que permite construir gráficos capa por capa. Esto facilita
la creación de gráficos complejos y altamente personalizados. Las capas
principales incluyen:

\begin{itemize}
\item
  \texttt{ggplot()}: El lienzo inicial.
\item
  \texttt{aes()}: Mapeo de variables a estéticas (ejes, color, forma).
\item
  Geoms: Las representaciones visuales de los datos (puntos, barras,
  líneas).
\end{itemize}
\end{frame}

\begin{frame}[fragile]{Dispersión con ggplot2}
\phantomsection\label{dispersiuxf3n-con-ggplot2}
\begin{Shaded}
\begin{Highlighting}[]
\FunctionTok{library}\NormalTok{(ggplot2)}

\NormalTok{plot\_dispersion }\OtherTok{\textless{}{-}}\NormalTok{ enfct\_2019 }\SpecialCharTok{\%\textgreater{}\%} 
  \FunctionTok{ggplot}\NormalTok{(}\FunctionTok{aes}\NormalTok{(}\AttributeTok{x =}\NormalTok{ edad, }\AttributeTok{y =}\NormalTok{ sueldo\_bruto\_ap\_monto)) }\SpecialCharTok{+}
  \FunctionTok{geom\_point}\NormalTok{() }\SpecialCharTok{+}
  \FunctionTok{labs}\NormalTok{(}\AttributeTok{title =} \StringTok{"Gráfico de Dispersión: Edad vs Sueldo Bruto"}\NormalTok{,}
       \AttributeTok{x =} \StringTok{"Edad"}\NormalTok{,}
       \AttributeTok{y =} \StringTok{"Sueldo Bruto"}\NormalTok{) }\SpecialCharTok{+}
  \FunctionTok{theme\_minimal}\NormalTok{()}
\end{Highlighting}
\end{Shaded}
\end{frame}

\begin{frame}[fragile]{Dispersión con ggplot2}
\phantomsection\label{dispersiuxf3n-con-ggplot2-1}
\begin{Shaded}
\begin{Highlighting}[]
\NormalTok{plot\_dispersion}
\end{Highlighting}
\end{Shaded}

\begin{center}\includegraphics{presentacion-modulo1_files/figure-beamer/unnamed-chunk-41-1} \end{center}
\end{frame}

\begin{frame}[fragile]{Histograma con ggplot2}
\phantomsection\label{histograma-con-ggplot2}
\begin{Shaded}
\begin{Highlighting}[]
\NormalTok{plot\_histograma }\OtherTok{\textless{}{-}}\NormalTok{ enfct\_2019 }\SpecialCharTok{\%\textgreater{}\%}
  \FunctionTok{ggplot}\NormalTok{(}\FunctionTok{aes}\NormalTok{(}\AttributeTok{x =}\NormalTok{ edad)) }\SpecialCharTok{+}
  \FunctionTok{geom\_histogram}\NormalTok{(}
    \AttributeTok{binwidth =} \DecValTok{5}\NormalTok{, }\AttributeTok{fill =} \StringTok{"blue"}\NormalTok{, }
    \AttributeTok{color =} \StringTok{"black"}\NormalTok{, }\AttributeTok{alpha =} \FloatTok{0.7}\NormalTok{) }\SpecialCharTok{+}
  \FunctionTok{labs}\NormalTok{(}\AttributeTok{title =} \StringTok{"Histograma de Edad"}\NormalTok{,}
       \AttributeTok{x =} \StringTok{"Edad"}\NormalTok{,}
       \AttributeTok{y =} \StringTok{"Frecuencia"}\NormalTok{) }\SpecialCharTok{+}
  \FunctionTok{theme\_minimal}\NormalTok{()}
\end{Highlighting}
\end{Shaded}
\end{frame}

\begin{frame}[fragile]{Histograma con ggplot2}
\phantomsection\label{histograma-con-ggplot2-1}
\begin{Shaded}
\begin{Highlighting}[]
\NormalTok{plot\_histograma}
\end{Highlighting}
\end{Shaded}

\begin{center}\includegraphics{presentacion-modulo1_files/figure-beamer/unnamed-chunk-43-1} \end{center}
\end{frame}

\begin{frame}[fragile]{Gráfico de boxplot}
\phantomsection\label{gruxe1fico-de-boxplot}
\begin{Shaded}
\begin{Highlighting}[]
\NormalTok{plot\_boxplot }\OtherTok{\textless{}{-}}\NormalTok{ enfct\_2019 }\SpecialCharTok{\%\textgreater{}\%}
  \FunctionTok{ggplot}\NormalTok{(}\FunctionTok{aes}\NormalTok{(}\AttributeTok{x =} \FunctionTok{factor}\NormalTok{(sexo), }\AttributeTok{y =}\NormalTok{ sueldo\_bruto\_ap\_monto)) }\SpecialCharTok{+}
  \FunctionTok{geom\_boxplot}\NormalTok{(}\AttributeTok{outliers =} \ConstantTok{FALSE}\NormalTok{) }\SpecialCharTok{+}
  \FunctionTok{labs}\NormalTok{(}\AttributeTok{title =} \StringTok{"Boxplot de Sueldo Bruto por Sexo"}\NormalTok{,}
       \AttributeTok{x =} \StringTok{"Mujer"}\NormalTok{,}
       \AttributeTok{y =} \StringTok{"Sueldo Bruto"}\NormalTok{) }\SpecialCharTok{+}
  \FunctionTok{theme\_minimal}\NormalTok{()}
\end{Highlighting}
\end{Shaded}
\end{frame}

\begin{frame}[fragile]{Gráfico de boxplot}
\phantomsection\label{gruxe1fico-de-boxplot-1}
\begin{Shaded}
\begin{Highlighting}[]
\NormalTok{plot\_boxplot}
\end{Highlighting}
\end{Shaded}

\begin{center}\includegraphics{presentacion-modulo1_files/figure-beamer/unnamed-chunk-45-1} \end{center}
\end{frame}

\section{Ejercicios prácticos}\label{ejercicios-pruxe1cticos-1}

\begin{frame}{Ejercicios}
\phantomsection\label{ejercicios-2}
\begin{enumerate}
\item
  Calcula los estadisticos aprendidos aquí para el ingreso de los
  individuos, edad.
\item
  Realiza el ejercicio anterior, pero ahora segmentando por sexo, y
  luego por estado civil.
\item
  Realiza un gráfico de dispersión entre edad y sueldo bruto, coloreando
  por sexo.
\item
  Calcula el promedio de sueldo bruto por provincia y realiza un gráfico
  de barras con los resultados. Sientete libre en personalizar el
  gráfico.
\item
  ¿Los ingresos más bajo a que provincia pertenecen?
\item
  ¿Son los solteros más joven que los casados?
\item
  ¿Los hombres ganan más que las mujeres?
\end{enumerate}
\end{frame}

\section{FINAL}\label{final}

\begin{frame}{Referencia}
\phantomsection\label{referencia}
Mendoza Vega, J. B. (2020). \emph{R para principiantes: Introducción al
análisis de datos con R y RStudio} (4a ed.). Editorial Académica
Española. \url{https://bookdown.org/jboscomendoza/r-principiantes4/}

Wickham, H., \& Grolemund, G. (2017). \emph{R para ciencia de datos} (1a
ed.). Ediciones Díaz de Santos. \url{https://r4ds.hadley.nz/index.html}

Villarroel-Riquelme, F. (2020). \emph{Introducción a la Estadística con
R y RStudio}. \url{https://rpubs.com/franciscovillarroel/estadistica_r}

James, G., Witten, D., Hastie, T., \& Tibshirani, R. (2013). \emph{An
Introduction to Statistical Learning: with Applications in R}. Springer.
\url{https://www.econometrics-with-r.org/1-introduction.html}
\end{frame}

\end{document}
